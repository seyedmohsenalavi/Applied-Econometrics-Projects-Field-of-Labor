\documentclass[12pt, a4paper]{article}
\usepackage[bottom=1in,top=1.38in, left=1in, right=1in, centering]{geometry}
\usepackage{amsmath}
\usepackage{fancyhdr}
\usepackage{comment}
\usepackage[margin=0pt,font=small,labelfont={bf,normalsize},labelsep=space]{caption}
\pagestyle{headings}
\lhead{\thepage}
\fancyfoot{}
\rhead{\leftmark}
\usepackage{hyperref}
\usepackage{graphicx}
\linespread{1.5}
\usepackage[round]{natbib}
\usepackage[localise=on]{xepersian}
\settextfont[Scale=1.05]{XB Niloofar}
\usepackage{xr}
\usepackage{longtable}
\usepackage{booktabs}
\usepackage{adjustbox}
%\usepackage{lscape}

\begin{document}
	\title{بررسی عوامل موثر بر ورود و عدم ورود به بازار کار افراد بالای 18 سال خانوار بر اساس ویژگی‌های این افراد طبق
داده‌های طرح آمارگیری نیروی کار}
	\author{سید محسن علوی}
	\date{\today}
	\maketitle
\newpage
	\tableofcontents
\newpage
	\listoffigures
	\listoftables
	
\newpage

\section{مقدمه}
در همه جوامع برای دولت‌ها اهمیت دارد تا بتوانند مشارکت افراد جامعه در بازار کار را افزایش دهند؛ برای این هدف بایستی دانست چه عواملی اثرگذار بر مشارکت افراد در بازار کار است تا بتوان سیاست‌هایی در جهت تغییر این عوامل به سمت افزایش مشارکت افراد در بازار کار طراحی کرد؛ در پژوهش ذیل سعی شده است در ابتدا به مرور مختصری از ادبیات این حوزه بپردازد تا مجموعه‌‌ای از عواملی که موثر بر مشارکت افراد در بازار کار بوده‌اند پدید آورد؛ بعد از آن به توصیف داده‌ طرح آمارگیری نیروی‌کار پرداخته می‌شود که سعی شده است که بررسی عوامل موثر بر مشارکت با استفاده از داده‌های این طرح آمارگیری انجام شود؛ بعد از آن، تصریح شناسایی عّلی پژوهش طرح گردیده است و سعی شده است تمامی اجزای آن توصیف گردد؛ در ادامه نتایج حاصل از آن تشریخ شده است و در انتها نتیجه‌گیری پژوهش و توصیه‌های سیاستی مبنتی بر آن اعلام گردیده است.
	\subsection{مرور ادبیات}
در ادامه سعی شده است مقالاتی که به ادبیات موضوع مرتبطند مورد بررسی قرار گیرند:
\begin{itemize}
\item
تیونگسون و یمتسوف(۲۰۰۸)
\LTRfootnote{\lr{Tiongson and Ruslan ,2008}}
با استفاده داده‌های بازار کار بوسنی هرزگوین طی سال‌های ۲۰۰۱-۲۰۰۴ و روش مدل‌سازی لاجیت
\LTRfootnote{\lr{Logit}}
سعی کرده‌اند عوامل موثر بر جابه‌جایی افراد در بازار کار  بین سه  گروه غیرفعال، بیکار و شاغل را پیدا کند؛ در این بررسی نتایج این مقاله نشان می‌دهد که به طور خاص در مورد جابه‌جایی افراد از گروه غیرفعال به دو گروه دیگر که موضوع مورد مطالعه پژوهش حاضر است سن، جنسیت، وضعت تاهل، تحصیلات و موقعیت جغرافیایی عواملی هستند که به طور معنی‌داری بر این جابه‌جایی اثرگذارند. مردان با احتمال بیشتری نسبت به زنان از گروه غیرفعال به گروه فعال جابه‌جا می‌شوند و نیز افزایش سن تا ۵۵ سالگی اثر مثبت بر این احتمال داشته و بعد از آن اثر سن برعکس می‌شود. افزایش سطح تحصیلات نیز به طور کلی اثر مثبت بر افزایش این احتمال دارد. در افراد مجرد نیز نسبت این احتمال به طور معنی‌داری بیشتر از متاهلین است.
\item
کوکه و اسپیرز(۲۰۰۵)
\LTRfootnote{\lr{Cooke and Speirs ,2005}}
 با استفاده از داده‌های سربازان حاضر در سرشماری ۱۹۹۰ آمریکا و روش مدل‌سازی لاجیت سعی کرده‌اند اثر مهاجرت در اثر سربازی را بر احتمال اشتغال افراد بعد از سربازی را بررسی کند. نتیجه اصلی مقاله ناین است که مهاجرت در اثر سربازی  فارغ از جنسیت احتمال اشتغال را ۱۰ درصد کاهش می‌دهد. از نتایج فرعی این مقاله که با موضوع مطالعه حاضر مرتبط است این است که این احتمال برای زنان نسبت به مردان به صورت معنی‌داری کمتر است؛ همچنین داشتن تحصیلات دانشگاهی و افزایش سن این احتمال را به صورت معنی‌دار افزایش می‌دهد.
 \item
 بوکفسکی و لواندوفسکی(۲۰۰۵) 
 \LTRfootnote{\lr{Bukowski and Lewandowski ,2005}}
 با استفاده از داده طرح آمارگیری نیروی‌کار لهستان طی سال‌های ۱۹۹۷-۲۰۰۴ و روش مدل‌سازی لاجیت با انتخاب‌های چندگانه سعی کرده است انتقال افراد از گروه بیکار به دو گروه شاغل و غیرفعال را طی این دوره زمانی با توجه به ویژگی‌های شخص افراد مانند جنسیت، سن، وضعیت تاهل، سطح تحصیلات، منطقه زندگی(شهری و روستایی) توضیح دهد. مقاله  در مورد سن توضیح می‌دهد  از طرفی افراد پیرتر محتمل ‌تر است که بیکار بمانند اما  جوان‌ترها  ورود به بازار کار را برای تحصیلات بیشتر به تاخیر می‌اندازند؛ از طرف دیگر جوان‌ترها با خطر عدم اشتغال مداوم مواجه شوند؛ درنتایج این مقاله نشان می‌دهد که زن بودن، مجرد بودن، افزایش سن، افزایش سطح تحصیلات، همگی باعث افزایش احتمال خروج فرد از بیکاری به گروه غیرفعالین یا شغالین می‌گردد.
 \item
 لاروا و ترل (۲۰۰۲) 
  \LTRfootnote{\lr{Lauerova and Terrell ,2002}}
 با استفاده از داده طرح آمارگیری نیروی کار جمهوری چک سعی کرده است اثر شکاف جنسیتی بر بیکاری را بسنجد؛ در این راستا با استفاده از مدل‌سازی روش لاجیت با انتخاب‌های چندگانه انتقال افراد از  یکی از سه حالت شاغل ،بیکار و غیرفعال  به دو گروه دیگر را بر متغیرهای توضیحی جنسیت، گروه سنی، سطح تحصیلات، اثر ثابت مکان و زمان بررسی می‌کند؛ نتایج مقاله نشان می‌دهد افراد جوان‌تر با احتمال بیشتری بیکار می‌شوند اما با احتمال بیشتر نیز شغل پیدا می‌کنند؛ پس نرخ بیکاری در جوانان بیشتر است اما دوره بیکاری ایشان نیز کوتاه‌تر است. در مورد سطح تحصیلات، افراد با تحصیلات کمتر با احتمال از گروه شاغل خارج می‌شوند و با احتمال کمتری به این گروه وارد می‌شوند؛ در مورد جنسیت نیز مردان با تحصیلات دانشگاهی  نسبت به زنان با تحصیلات دانشگاهی با احتمال بیشتری از بیکاری خارج می‌شوند؛  در مورد تاهل افراد مجرد با احتمال ییشتری نسبت به متاهلین بیکار می‌شوند؛ مرتبط‌ترین نتیجه مقاله به پژوهش حاضر این است که زنان با تحصیلات دانشگاهی و متوسط نسبت به مردان با همین سطح تحصیلات با احتمال بیشتری به بازار کار وارد می‌شوند. این مقاله نشان می‌دهد که سطح تحصیلات، سن و وضعیت تاهل ویژگی‌هایی هستند که بخشی از شکاف جنسیتی بیکاری را توضیح می‌دهند.
  \item
سارانی و دیگران(۱۳۹۳) با استفاده از داده هزینه-درآمد خانوار سال ۱۳۸۸ و روش مدل‌سازی لاجیت سعی کرده است عوامل موثر بر مشارکت زنان در بازار کار ایران را بررسی کند؛ متغیرهای توضیحی مورد استفاده در این مقاله که در داده طرح آمارگیری نیروی کار موجود است تحصیلات، سن و منطقه زندگی است؛ نتایج مقاله نشان می‌دهد که مهمترین عامل موثر بر مشارکت زنان سطح تحصیلات ایشان است.
   \item
 مشیری و دیگران(۱۳۹۴) با استفاده از داده‌های بودجه خانوار سال‌های ۶-۱۳۸۰ و روش مدل‌سازی لاجیت سعی کرده است عوامل موثر بر مشارت افراد در بازار کار ایران را بررسی کند؛ متغیرهای استفاده شده در این مقاله سن، تاهل، سطح تحصیلات و ... است که نتایج حاصل نشان می‌دهد احتمال مشارکت در بازار کار برای گروه‌های سنی میانی از همه بیشتر است؛ همچنین افزایش سطح تحصیلات احتمال مشارکت در بازار کار را مخصوصا برای زنان افزایش می‌دهد. تاهل برخلاف سایر مقالات ادبیات اثر مثبتی بر احتمال مشارکت افراد در بازار کار داشته است.
 \item
 نحوی و دیگران(۱۳۹۱) سعی کرده است با روش مدل‌سازی لاجیت عوامل موثر بر مشارکت زنان در شهر مشهد را بررسی کند؛ در این بررسی از متغیرهای سن، وضعیت تاهل، سطح تحصیلات و ... استفاده شده است که نتایج حاصل از آن نشان می‌دهد که متاهل بودن احتمال مشارکت زنان را کاهش و افزایش سطح تحصیلات این احتمال را افزایش می‌دهد و سن اثر معنی‌داری بر این احتمال نداشته است.
\end{itemize}
\section{داده}
داده‌ استفاده شده در این پژوهش داده های طرح آمارگیری نیروی کار ایران(
\lr{LFS}
) طی سال‌های ۸-۱۳۹۲  است. متغیرهای موجود در این داده به شرح ذیل است:
\subsection{سن}
در جدول 
\ref{table_age}
اطلاعات آماری سن افراد در این داده قابل مشاهده است.
\begin{table}
	\centering
	\caption{ویژگی‌های آماری سن افراد طی سال‌های ۸-۱۳۹۲}\label{table_age}
	\begin{latin}
		\begin{LTR}
			\input{age.tex}
		\end{LTR}
	\end{latin}

\end{table}
همچنین در شکل
\ref{fig_pp}،
هرم جمعیتی در سال اول و آخر داده‌ها نشان می‌دهد که جمعیت در سن کار پیرتر شده است. این مشاهده مطابق جدول
\ref{table_age}
است که مقادیر میانگین و میانه داده طی این سال‌ها افزایش یافته است.
با توجه به موضوع پژوهش افراد در سن کار مورد توجه بوده است پس از داده افراد بین سن ۱۸ تا ۶۵ سال نگه داشته شد و باقی افراد از داده حذف شدند.  در ادامه تمامی خلاصه‌های آماری، جداول و شکل‌ها برای افراد در سن کار تهیه شده است. 
\begin{figure}
	\centering
	\includegraphics[width=\textwidth,keepaspectratio]{combine2.png}
	\caption{هرم جمعیتی داده در دو سال ۱۳۹۲ و ۱۳۹۸}\label{fig_pp}
\end{figure}
\subsection{سطح تحصیلات و وضعیت تاهل}
در شکل
\ref{fig_com}
سهم افراد در داده از نظر سطح تحصیلات و وضعیت تاهل قابل مشاهده است. عمده افراد سطح تحصیلات متوسطه را دارا هستند(۵۵٪)و نیز متاهلند(۷۰٪).
\begin{figure}
	\centering
	\includegraphics[width=\textwidth,keepaspectratio]{combine.png}
	\caption{سهم افراد در داده از نظر سطح تحصیلات و وضعیت زناشویی}\label{fig_com}
\end{figure}

\subsection{سایر اطلاعات}
در جدول
  \ref*{table_some}
  اطلاعاتی  به طور خلاصه در مورد سایر متغیرهایی که در داده وجود دارند قابل مشاهده است: نسبت زن به مرد، نسبت شهری به روستایی، نسبت محصلین به کل، نسبت متاهلین به کل، نسبت ملیت ایرانی به کل.
\begin{table}
	\centering
	\caption{نسبت مرد به زن، نسبت داده روستایی به کل، نسبت افراد در حال تحصیل به کل و نسبت افراد متاهل به کل، نسبت ایرانیان به کل طی سال‌های ۸-۱۳۹۲}\label{table_some}
	\begin{latin}
		\begin{LTR}
			\input{table1.tex}
		\end{LTR}
	\end{latin}
\end{table}

\subsection{مشارکت در بازار کار}
وضعیت افراد در سن کار از نظر فعالیت سه حالت مختلف دارد: شاغل، بیکار و غیرفعال؛ دو گروه شاغل و بیکار فعالین بازار کار هستند که به بازار کار وارد شده‌اند و گروه غیرفعال از بازار کار به کلی خارج شده‌اند. نرخ مشارکت نسبت گروه فعال به کل افراد در سن کار را نشان می‌دهد. متغیر 
\lr{pr}
ساخته شده در کد متغیر مجازی 
\LTRfootnote{\lr{Dummy variable}}
است که برای گروه فعال یک و برای گروه غیرفعال صفر اخذ می‌کند. در واقع نرخ مشارکت میانگین این متغیر را نشان می‌دهد.
 روند زمانی نرخ مشارکت طی این سال‌ها برای گروه‌های با سطح تحصیلات مختلف در شکل
 \ref{fig_edu}
 قابل مشاهده است. واضح است که نرخ مشارکت گروه‌ با تحصیلات دانشگاهی طی این سال‌ها رشد بیشتری نسبت به سایر گروه‌ها داشته است. از طرف دیگر به نظر می‌رسد نرخ مشارکت گروه بی‌سوادان تفاوت جدی با سه گروه دیگر دارد و همبستگی افزایش سطح تحصیلات و نرخ مشارکت در نمودار قابل مشاهده است.
\begin{figure}
		\centering
	\includegraphics[width=\textwidth,keepaspectratio]{g1.png}
	\caption{نرخ مشارکت طی سال‌های ۸-۱۳۹۲ برای گروه‌های با سطح تحصیلات مختلف}\label{fig_edu}
\end{figure}
	همچنین در شکل
	\ref{fig_sex}
	روند نرخ مشارکت برای دو گروه مرد و زن رسم شده است؛ به نظر می‌رسد رشد نرخ مشارکت  طی این سال‌ها برای گروه مردان بیشتر از گروه زنان بوده است که شکاف جنسیتی مشارکت در بازار کار به وضوح در شکل قابل مشاهده است.
\begin{figure}
	\centering
	\includegraphics[width=\textwidth,keepaspectratio]{g2.png}
	\caption{نرخ مشارکت مرد و زن طی سال‌های ۸-۱۳۹۲}\label{fig_sex}
\end{figure}
\newpage \clearpage
\section{نتایج}
\subsection{شناسایی علّی}
تصریح اصلی شناسایی علّی عوامل موثر بر ورود و عدم ورود به بازار کار معادله 
\ref{eq1}
 است.

\begin{equation}
	\label{eq1}
P(pr_t=1)=\beta_0 + \beta_1 sex+ \beta_2 age_t + \beta_3 age_t^2 + \beta_4 educ_t+ \beta_5 edu_t +\beta_6 citizen + \beta_7 mar_t + \gamma_t + \gamma_r+\epsilon
\end{equation}
متغیر وابسته تصریح احتمال ورود فرد به بازار کار است. متغیر گسسته $age$ سن افراد را نشان می‌دهد که  درجه دو آن هم در تصریح رگرسیون استفاده شده است؛ چراکه انتظار می‌رود در سن‌های پایین بدلیل بی‌تجربگی با افزایش سن و تجربه  و همچنین تشکیل خانواده احتمال ورود به بازار کار برای فرد افزایش یابد و بدر سن‌های بالا به علت کهولت سن و بازدهی کمتر احتمال وجود به بازار کار برای فرد کاهش یابد. متغیر $educ$ متغیر مجازی است و اینکه فرد محصل است یا خیر را نشان می‌دهد. محصل نبودن  به عنوان حالت مبنای این متغیر در جداول رگرسیون انتخاب شده است. متغیر $edu$ متغیر مجازی دیگری است که سطح تحصیلات فرد را در چهار دسته. بی‌سواد، تحصیلات ابتدایی، تحصیلات متوسطه و تحصیلات دانشگاهی نشان می‌دهد.  حالت مببنا در جدول رگرسیون  برای این متغیر بی‌سوادی است. متغیر مجازی $citizen$ تابعیت افراد را نشان می‌دهد؛ این متغیر سه دسته ایرانی، افغان و سایر کشورها را داراست و در جدول رگرسیون حالت ایرانی آن مبناست. متغیر مجازی $mar$ چهار حالت دارد: متاهل(حالت مبنا در جداول رگرسیون)، بیوه، مطلقه و مجرد؛ متغیر $\gamma_t$ اثر ثابت زمان(سال و فصل) و متغیر $\gamma_r$ اثر ثابت منطقه(استان و بخش روستایی/شهری) را نشان میدهند. اندیس $t$ نشان‌دهنده زمان و اندیس $r$ نشان‌دهنده منطقه است.
\subsection{خروجی رگرسیون}
جدول
\ref{reg}
رگرسیون‌‌های مختلف ایجاد شده را نشان می‌دهد؛ متغیرهای مستقل دستون اول رگرسیون جنسیت و سن(و توان درجه دو آن) هستند؛ در ستون دوم سطح تحصیلات و محصل بودن فرد و در ستون سوم و چهارم اثر ثابت زمان و منطقه نیز  به متغیرهای مستقل رگرسیون اضافه شده‌اند. در ستون پنجم که تصریح منتخب و کامل  در بین رگرسیون‌هاست دو متغیر مجازی تابعیت و وضعیت تاهل افراد به متغیرهای مستقل رگرسیون اضافه شده است. مقایسه ضرایب ستون‌های یک تا چهار جدول نشان می‌دهد که نتایج نسبت به افزودن متغیرهای جدید پایدار است و علامت ضرایب در ستون‌های مختلف تغییر نکرده است.
\newpage
\begin{latin}
\begin{center}
	\begin{LTR}
		\input{reg.tex}
	\end{LTR}
\end{center}
}
\end{latin}
در جدول
\ref{reg2}
سعی شده است تا سه روش برای رگرسیون متغیر وابسته بر متغیرهای مستقل استفاده شود تا با یکدیگر مقایسه شوند و پایداری نتایج روش لاجیت با دو روش دیگر آزمون گردد؛
\LTRfootnote{\lr{Robustness check}}
 در ستون اول، متغیر احتمال فعال بودن فرد ِدر سن کار روی همان متغیرهای مستقل با روش حداقل مرابعات معمولی
\LTRfootnote{\lr{OLS}}
 رگرسیون شده است؛ در ستون دوم به جای روش لاجیت
 از روش پروبیت
 \LTRfootnote{\lr{Probit}}
 استفاده شده است. مقایسه جهت ضرایب دو ستون اول و دوم جدول
 \ref{reg2}
 با ستون ۴ جدول
  \ref{reg}
  نشان‌دهنده پایداری نتایج است.
 ستون سوم این جدول همان ستون ۴ جدول
\ref{reg}
است با این تفاوت که روی خانوار خوشه‌بندی
 \LTRfootnote{\lr{Clustering}}
شده است؛ دلیل استفاده از این تکنیک این بوده است که می‌دانیم ویژگی‌های یک خانواده و اعضای آن در طول زمان از طرق مختلف بر مشارکت افراد در بازار کار اثر گذار است و بایستی این مورد نیز کنترل شود؛ با کنترل این مورد نیز جهت تمامی ضرایب ثابت مانده است که نشان از پایداری نتایج ستون ۴ جدول
  \ref{reg}
دارد. 
در ستون ۴ روش مدل‌سازی لاجیت است و فقط داده‌ها به زنان شهری محدود شده است؛  مقایسه این ستون با ستون ۴ جدول
  \ref{reg}
  و برابری جهت ضرایب در این دو ستون نشان‌دهنده این است که مردان شهری و زنان شهری تفاوتی در جهت اثربخشی عوامل مختلف ندارند. 
 در ستون ۵ این جدول داده‌ها به مردان روستایی محدود شده است؛ مقایسه این ستون با ستون ۴ جدول
\ref{reg}
 امکان مقایسه نتایج در برای مردان شهری و روستایی فراهم شده است؛ جهت ضرایب در این دو در مواردی مانند سن، محصل بودن، تابعیت همراستاست؛ اما در دو مورد سطح تحصیلات و وضعیت تاهل تفاوت دارد که در ادامه توضیح داده خواهد شد.

\begin{latin}
	\begin{center}
		\begin{LTR}
			\input{reg2.tex}
		\end{LTR}
	\end{center}
}
\end{latin}

نتایج ستون ۴ جدول
\ref{reg}
و ستون ۴ و ۵ جدول 
\ref{reg2}
به شرح ذیل است:
\begin{itemize}
	\item 
مطابق تیونگسون و یمتسوف(۲۰۰۸)، کوکه و اسپیرز(۲۰۰۵) و  لاروا و ترل (۲۰۰۲) ، احتمال ورود مردان به بازار کار به صورت معنی‌داری بیشتر از زنان است.
	\item 
مطابق تیونگسون و یمتسوف(۲۰۰۸) و نحوی و دیگران(۱۳۹۱)، با افزایش سن احتمال ورود افراد به بازار کار بیشتر می‌شود اما چون علامت ضریب سن به توان دو منفی شده است، واضح است که همبستگی احتمال ورود به بازار کار با سن سهمی رو به پایین است؛ یعنی در سن‌های پایین افزایش سن احتمال ورود افراد به بازار کار را افزایش می‌دهد اما در سن‌های بالا برعکس است و افزایش سن باعث کاهش احتمال ورود افراد به بازار کار می‌شود. توجه به این نتیجه و نیز افزایش میانه و میانگین سن مطابق جدول 
\ref{table_age}
و نیز چاق شدن شکم هرم جمعیتی از سال ۱۳۹۲ به سال ۱۳۹۸ در شکل 
\ref{fig_pp}
نتیجه مهمی را حاصل می‌کند و آن اینکه یکی از دلایل مهم افزایش نرخ مشارکت در طی سال‌های ۱۳۹۲ تا ۱۳۹۸ افزایش تعداد افراد در سنین میانی بوده است که در این گروه‌ها نرخ مشارکت بیشترین مقدار را داراست؛ دلیل این مسئله نیز در  بررسی مقاله بوکفسکی و لواندوفسکی(۲۰۰۵) در بخش مرور ادبیات اشاره شد.
	\item 
مطابق لاروا و ترل (۲۰۰۲)، بوکفسکی و لواندوفسکی(۲۰۰۵)، کوکه و اسپیرز(۲۰۰۵) و سارانی و دیگران(۱۳۹۳) در مورد سطح تحصیلات، در مناطق شهری متوسط احتمال ورود به بازار کار برای افراد بی‌سواد از دو گروه افراد با تحصیلات دبستان و متوسطه بیشتر و از افراد با تحصیلات دانشگاهی کمتر است. اما در مناطق روستایی هرچه سطح تحصیلات افراد افزایش یابد این احتمال افزایش می‌یابد.
		\item 
مطابق تیونگسون و یمتسوف(۲۰۰۸) و نحوی و دیگران(۱۳۹۱) در مورد وضعیت تاهل در مناطق شهری، احتمال ورود به بازار کار یا متوسط نرخ مشارکت برای افراد مجرد، بیوه و مطلقه از متاهلین بیشتر است؛ این نتیجه مطابق شهود برای زنان است چراکه انتظار داریم زنان در اثر تاهل با مرد شاغل انگیزه کمتری برای مشارکت در بازار کار دارند؛ این درحالی است که طلاق و فوت شوهر باعث می‌شود ایشان خود مجبور به مشارکت در بازار کار شوند. برای مردان نیز در صورت طلاق و فوت همسر انتظار داریم فرد برای تامین هزینه‌های زندگی خانواده مشارکت بیشتری در بازار کار داشته باشد؛ در مورد مردان مجرد ضریب خلاف شهود اما مطابق نتایج مقالاتی که مطرح شد است. در مناطق روستایی مردان متاهل با احتمال بیشتری نسبت به سایر وضعیت‌ها در بازار کار مشارکت می‌کنند که مطابق شهود است.
	\item 
متوسط احتمال ورود به بازار کار برای مردان افغان‌ بیشتر از مردان ایرانی‌ است ولی این احتمال برای زنان افغان کمتر از زنان ایرانی است که مطابق انتظار است؛ این احتمال برای سایر تابعیت‌ها از ایرانی‌ها کمتر است.
		\item 
محصل بودن افراد در تمامی مناطق و با هر جنسیتی باعث کاهش احتمال ورود ایشان به بازار کار می‌گردد که مطابق انتظار است فردی که مشغول تحصیل است فرصت عرضه نیروی کار به بازار کار همزمان با تحصیلا را کمتر دارد.
\end{itemize}

اندازه اثر حاشیه‌ای
 \LTRfootnote{\lr{Marginal effects}}
 هر یک از متغیرهای مستقل توضیحی بر احتمال ورود به بازار کار  در ستون ۵ جدول 
\ref{reg}
آمده است؛ سطح پایه متغیرهای مجازی این ستون مرد بی‌سواد غیرمحصل ایرانی متاهل است و در نتایج ذیل کاربرد لفظ فرد اشاره به فردی با این ویژگی‌ها دارد؛ این نتایج به شرح ذیل است:
\begin{itemize}
	\item
در ثبات سایر متغیرهای توضیحی، به طور متوسط افراد ۵۹.۵ واحد درصد با احتمال بیشتر نسبت به زنان وارد بازار کار می‌‌شوند.
	\item
در ثبات سایر متغیرهای توضیحی، به طور متوسط هر سال افزایش سن در افراد ۴ واحد درصد احتمال ورود به بازار کار را افزایش می‌دهد. این در حالیست که افزایش توان دو سن در ثبات سایر متغیرها به طور متوسط ۰.۱ واحد درصد احتمال ورود به بازار کار را کاهش می‌دهد.
	\item
در ثبات سایر متغیرهای توضیحی، به طور متوسط محصل بودن افراد ۳۸.۱ واحد درصد احتمال ورود به بازار کار ایشان را کاهش می‌دهد.
	\item
	در ثبات سایر متغیرهای توضیحی، به طور متوسط افراد با تحصیلات دانشگاهی نسبت به افراد بی‌سواد ۱۵.۷ درصد احتمال بیشتری دارد که به بازار کار وارد شوند. این در حالیست که احتمال ورود به بازار کار زنان با تحصیلات دبستان و متوسطه تفاوت بسیار اندکی(کمتر از نیم واحد درصد) با زنان بی‌سواد دارد؛ این نتیجه نشان از این دارد که سطح تحصیلات افراد شهری(چه مرد و چه زن
 \footnote{
 	خروجی
 	\lr{marginal effects}
 	برای زنان برای طولانی نشدن متن در متن گزارش نشده است ولی نتایج زنان و مردان شهری در این متغیر برابر بوده است.
 	}
) تا به سطح دانشگاهی نرسد، تفاوتی در احتمال ورود ایشان به بازار کار ایجاد نمی‌شود.
	\item
در ثبات سایر متغیرهای توضیحی به طور متوسط، ورود مردان اففان به بازار کار ۴.۶ واحد درصد احتمال بیشتری نسبت به مردان ایرانی دارد. این احتمال برای سایر تابعیت‌ها ۷.۶ واحد درصد کمتر از ایرانی‌هاست.
	\item
	در ثبات سایر متغیرهای توضیحی به طور متوسط، ورود افراد متاهل به ترتیب نسبت به افراد بیوه، طلاق گرفته و مجرد به بازار کار ۲، ۱۰.۵ و ۲.۵ واحد درصد کمتر است.
\end{itemize}
\newpage \clearpage
\section{نتیجه‌گیری}
سعی شد  با استفاده از داده‌های طرح آمارگیری نیروی‌کار در سال‌های ۸-۱۳۹۲ و روش لاجیت عوامل موثر بر احتمال مشارکت افراد در بازار کار ایران بررسی گردد؛ نتایج پژوهش حاضر نشان داد که تمامی متغیرهای استفاده شده در تصریح شناسایی عّلی  از جمله سن،  جنسیت، سطح تحصیلات، وضعیت تاهل، تابعیت، و وضعیت محصیل بودن به صورت معنی‌داری بر احتمال مشارکت افراد در بازار کار اثر داشتند که نتایج حاصل مطابق ادبیات بررسی شده و تاییدی بر آن‌ها و در اکثر موارد مطابق شهود بود.

مطابق نتایج بدست آمده زنان با تحصیلات دانشگاهی نسبت به سایر زنان با سطح تحصیلات پایین‌تر حدود ۱۶ واحد درصد با احتمال بیشتری به بازار کار وارد می‌شوند؛ با توجه به شکاف جنسیتی در نرخ مشارکت ایران توصیه سیاستی از این جهت چنین می‌تواند باشد که شرایط تحصیل زنان در دانشگاه تسهیل گردد چراکه تغییر سطح تحصیلات ایشان از سطح متوسطه به دانشگاهی می‌تواند اثر زیادی بر ورود ایشان به بازار کار داشته باشد. همچنین تفاوتی بین اثر سطح تحصیلات مردان در مناطق شهری و روستایی وجود داشت بدین صورت که در مناطق شهری افزایش سطح تحصیلات تا قبل از دانشگاه اثر زیادی بر مشارکت نداشت در حالی ‌که در مناطق روستایی افزایش سطح تحصیلات از بی‌سوادی به باسوادی اثر زیادی دارد و در واقع این مسئله مطابق شهود است که در مناطق روستایی باتوجه به مشاغل روستایی سطوح پایین‌تری از تحصیلات نسبت به شهر نیاز است.

توجه به نتایج اشاره شده در مورد سن افراد نیز باعث می‌شود سیاست‌گذار بداند که یکی از دلایل افزایش نرخ مشارکت طی سال‌های ۸-۱۳۹۲ افزایش طبیعی تعداد گروه سنی میانی در اثر رسیدن متولدین دهه شصت بدین سنین بوده است و باید بداند مطابق انتظار از ادبیات با گذر زمان و پیر شدن این متولدین از سنین میانی به سنین پیری نرخ مشارکت این گروه که اثر زیادی در بازار کار دارند کاهش خواهد یافت.
\newpage \clearpage
\section{منابع}
\indent
\hspace{0.5cm}
سارانی, ز., کشته گر, ب., & کشاورز حداد, غ. (1393). مشارکت زنان متاهل در بازار کار ایران: مدلسازي غیرخطی تابع لاجیت. 115-134.

مشیری, س., طایی, ح., & پاشازاده, ح. (۱۳۹۴). عوامل مؤثر بر نرخ مشارکت نیروی کار در بازار کار ایران.

نحوی, ا., & قربانی, م. (۱۳۹۱). بررسی عوامل مؤثر بر مشارکت زنان در بازار کار )مطالعه موردی شهرمشهد(. 147-158.

\begin{latin}
\begin{LTR}
Bukowski, M., & Lewandowski, P. (2005). Transitions from unemployment in Poland: a multinomial logit analysis. Labor and Demography, 511008.


\indent
Cooke, T. J., & Speirs, K. (2005). Migration and employment among the civilian spouses of military personnel. Social Science Quarterly, 86(2), 343-355.
\indent


Tiongson, E., & Yemtsov, R. Ruslan (2008): Bosnia and Herzegovina 2001-2004: Enterprise Restructuring, Labor Market Transitions and Poverty. World Bank Policy Research Working Paper No, 4479.


\indent
Stefanova Lauerova, J., & Terrell, K. (2002). Explaining gender differences in unemployment with micro data on flows in post-communist economies. Available at SSRN 341545.
\end{LTR}
\end{latin}
\end{document}